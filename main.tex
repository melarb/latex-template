% -----------------------------------------------------------------------------
% LaTeX Template created by Muhammad Elarbi
% Date: May 21, 2024
% Description: This template was originally created for use in the Electrical and Computer
% Engineering Department at Northeastern University. It has been adapted to serve as a
% general template for any institution.
% -----------------------------------------------------------------------------

\documentclass[12pt]{article}

% -----------------------------------------------------------------------------
% Packages used in this template. Modify as needed based on your requirements.
% -----------------------------------------------------------------------------
\usepackage[margin=1in]{geometry} % Adjust margins if needed
\usepackage{amsmath,amsfonts,amssymb} % For mathematical symbols and equations
\usepackage{graphicx} % For including images
\usepackage{fancyhdr} % For customizing headers and footers
\usepackage{enumitem} % For customizing lists
\usepackage{listings} % For including code listings
\usepackage{matlab-prettifier} % For formatting MATLAB code
\usepackage{hyperref} % For hyperlinks
\usepackage{float} % For [H] placement option for figures

% -----------------------------------------------------------------------------
% Header and Footer Customization. Modify these based on your institution's requirements.
% -----------------------------------------------------------------------------
\pagestyle{fancy} % Use the fancyhdr package to customize headers and footers
\fancyhf{} % Clear all header and footer fields
\lhead{University Name} % Change 'University Name' to the name of your university
\rhead{Department Name} % Change 'Department Name' to the name of your department
\lfoot{Course Code: Course Title} % Change 'Course Code: Course Title' to your course code and title
\rfoot{Page \thepage} % Right footer to display the current page number
\setlength{\parindent}{0pt} % Set paragraph indentation to zero
\setlength{\parskip}{1em} % Set the space between paragraphs

\begin{document}

% -----------------------------------------------------------------------------
% Title Page
% -----------------------------------------------------------------------------
\pagenumbering{gobble} % Suppress page numbering on the title page
\title{Assignment Title} % Change 'Assignment Title' to the title of your assignment
\author{Student Name(s)} % Change 'Student Name(s)' to your name(s)
\date{\today} % Use the current date. Change if needed.
\maketitle % Generate the title page

% -----------------------------------------------------------------------------
% This template was originally created by Muhammad Elarbi for use in the Electrical and
% Computer Engineering Department at Northeastern University. It has been adapted
% to serve as a general template for any institution. For any questions or suggestions,
% please contact elarbi.m@northeastern.edu.
% -----------------------------------------------------------------------------

% Comment out or delete the following line to remove the footnote before submission.
\footnotetext{\LaTeX~Template created by Muhammad Elarbi. For questions or suggestions, contact elarbi.m@northeastern.edu.}

% -----------------------------------------------------------------------------
% Class Information Section
% -----------------------------------------------------------------------------
\section*{Class Information}
\textbf{Class:} Course Code: Course Title \\ % Change 'Course Code: Course Title' to your course code and title
\textbf{Instructor:} Instructor Name \\ % Change 'Instructor Name' to the name of your instructor
\textbf{Semester:} Semester and Year \\ % Change 'Semester and Year' to the current semester and year
\textbf{Name(s):} Student Name(s) \\ % Change 'Student Name(s)' to your name(s)
\textbf{Group:} Group Number % Change 'Group Number' to your group number, if applicable

\newpage
\pagenumbering{arabic} % Start page numbering from 1
\setcounter{page}{1} % Set the page counter to 1
\tableofcontents % The Table of Contents will be generated automatically based on the sections and subsections in your document.

% -----------------------------------------------------------------------------
% Solutions Section
% -----------------------------------------------------------------------------
\newpage
\section*{Solutions}

% Problem 1 Section
% Modify the number of problems and structure as needed for your assignment.
\section*{Problem 1} % Modify the number of problems as needed
\addcontentsline{toc}{section}{Problem 1}
\subsection*{Problem 1, Part (a)} % Modify or add subsections as needed
\addcontentsline{toc}{subsection}{Problem 1, Part (a)}

% Insert your problem 1, part (a) solution here

\subsection*{Problem 1, Part (b)}
\addcontentsline{toc}{subsection}{Problem 1, Part (b)}

% Insert your problem 1, part (b) solution here

% Problem 2 Section
% Modify the number of problems and structure as needed for your assignment.
\newpage
\section*{Problem 2} % Modify the number of problems as needed
\addcontentsline{toc}{section}{Problem 2}
\subsection*{Problem 2, Part (a)}
\addcontentsline{toc}{subsection}{Problem 2, Part (a)}

% Insert your problem 2, part (a) solution here

\subsection*{Problem 2, Part (b)}
\addcontentsline{toc}{subsection}{Problem 2, Part (b)}

% Insert your problem 2, part (b) solution here

% Problem 3 Section
% Modify the number of problems and structure as needed for your assignment.
\newpage
\section*{Problem 3} % Modify the number of problems as needed
\addcontentsline{toc}{section}{Problem 3}
\subsection*{Problem 3, Part (a)}
\addcontentsline{toc}{subsection}{Problem 3, Part (a)}

% Insert your problem 3, part (a) solution here

\subsection*{Problem 3, Part (b)}
\addcontentsline{toc}{subsection}{Problem 3, Part (b)}

% Insert your problem 3, part (b) solution here

% Problem 4 Section
% Modify the number of problems and structure as needed for your assignment.
\newpage
\section*{Problem 4} % Modify the number of problems as needed
\addcontentsline{toc}{section}{Problem 4}
\subsection*{Problem 4, Part (a)}
\addcontentsline{toc}{subsection}{Problem 4, Part (a)}

% Insert your problem 4, part (a) solution here

\subsection*{Problem 4, Part (b)}
\addcontentsline{toc}{subsection}{Problem 4, Part (b)}

% Insert your problem 4, part (b) solution here

\subsection*{Problem 4, Part (c)}
\addcontentsline{toc}{subsection}{Problem 4, Part (c)}

% Insert your problem 4, part (c) solution here

\subsection*{Problem 4, Part (d)}
\addcontentsline{toc}{subsection}{Problem 4, Part (d)}

% Insert your problem 4, part (d) solution here

\end{document}
